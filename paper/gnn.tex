\documentclass{article}

\usepackage{hyperref}
\usepackage{url}
\usepackage{listings}
\usepackage{pifont}
\usepackage{tikz}
\usepackage{subfig}
\usepackage{multirow}
\usepackage{pgfplots}
\usepackage{booktabs}
\usepackage{dcolumn}
\usepackage{cancel}
\usepackage{ifmtarg}
\usepackage{verbatim}
\usepackage{graphicx}
\usepackage{chngpage}
\usepackage{xspace}
%\usepackage{algorithm}
\usepackage{algpseudocode}
\usepackage{balance}
\usepackage{amsmath}
\usepackage{amssymb}
\usepackage{amsthm}
\usepackage{cleveref}
\usepackage{threeparttable}

\usepackage{microtype}
\usepackage{booktabs} % for professional tables


% Attempt to make hyperref and algorithmic work together better:
\newcommand{\theHalgorithm}{\arabic{algorithm}} 
\newcommand{\er}[1]{\mbox{\rm\em #1}}
\newcommand{\sys}{LuxGNN\xspace}
\newcommand{\xg}{HAG\xspace}
\newcommand{\xgs}{HAGs\xspace}
\newcommand{\ZJ}[1] {\textcolor{blue}{[ZJ: #1]}}
\newcommand{\mw}[1] {\mathcal{\widehat{#1}}}
\newcommand{\m}[1] {\mathcal{#1}}
\newtheorem{theorem}{Theorem}
\DeclareMathOperator*{\argmax}{arg\,max}
\newcommand{\jure}[1]{{\color{magenta}[[#1]]}}
\newcommand{\hide}[1]{}

% Use the following line for the initial blind version submitted for review:
\usepackage{icml2019}

% If accepted, instead use the following line for the camera-ready submission:
%\usepackage[accepted]{icml2019}

% The \icmltitle you define below is probably too long as a header.
% Therefore, a short form for the running title is supplied here:

\icmltitlerunning{Redundancy-Free Computation Graphs for Graph Neural Networks}

\begin{document}

\twocolumn[
\icmltitle{Redundancy-Free Computation Graphs for Graph Neural Networks}

% It is OKAY to include author information, even for blind
% submissions: the style file will automatically remove it for you
% unless you've provided the [accepted] option to the icml2019
% package.

% List of affiliations: The first argument should be a (short)
% identifier you will use later to specify author affiliations
% Academic affiliations should list Department, University, City, Region, Country
% Industry affiliations should list Company, City, Region, Country

% You can specify symbols, otherwise they are numbered in order.
% Ideally, you should not use this facility. Affiliations will be numbered
% in order of appearance and this is the preferred way.
\icmlsetsymbol{equal}{*}

\begin{icmlauthorlist}
\icmlauthor{aaa}{to}
\end{icmlauthorlist}

\icmlaffiliation{to}{nowhere}

\icmlcorrespondingauthor{aaa}{c.vvvvv@googol.com}

% You may provide any keywords that you
% find helpful for describing your paper; these are used to populate
% the "keywords" metadata in the PDF but will not be shown in the document
\icmlkeywords{Machine Learning, ICML}

\vskip 0.3in
]

% this must go after the closing bracket ] following \twocolumn[ ...

% This command actually creates the footnote in the first column
% listing the affiliations and the copyright notice.
% The command takes one argument, which is text to display at the start of the footnote.
% The \icmlEqualContribution command is standard text for equal contribution.
% Remove it (just {}) if you do not need this facility.

\printAffiliationsAndNotice{}  % leave blank if no need to mention equal contribution
%\printAffiliationsAndNotice{\icmlEqualContribution} % otherwise use the standard text.

\begin{abstract}
Graph Neural Networks (GNNs) have revolutionized machine learning on graphs.
GNNs are based on repeated aggregations of information across a node's neighbors, and because many neighbors are shared between different nodes, this leads to many repeated and inefficient computations.
%means that many computations are repeated and inefficient.
%
We propose {\em \xg}, a new GNN graph representation that explicitly avoids repeated computations by managing intermediate aggregation results hierarchically, which reduces redundant operations and eliminates unnecessary data transfers in GNN training.
We introduce a cost model to quantitatively evaluate the runtime performance of different \xgs and use a novel \xg search algorithm to find optimized \xgs.
%We introduce a novel cost model and the develop an algorithm that provably finds the minimum-cost GNN computation graph.
%
Experiments show that the \xg representation significantly outperforms the standard GNN graph representation by increasing the end-to-end training throughput up to 2.8$\times$ and reducing the aggregations and data transfers in GNN training by up to 6.3$\times$ and 5.6$\times$, while maintaining the original model accuracy.
\hide{ %% Jure
Existing graph neural networks (GNNs) use ordinary graph representation that directly connects each vertex in a graph with its neighbors.
Each vertex computes its activations by aggregating its neighbors independently, resulting in redundant computation and unnecessary data transfers.
In this paper, we propose \xg, a new graph representation that explicitly manages intermediate aggregation results hierarchically, which reduces redundant computation and eliminates unnecessary data transfers.
We introduce a cost model to quantitatively evaluate the runtime performance of different \xgs and use novel graph search algorithm to find highly optimized \xgs under the cost model.
Our evaluation shows that the \xg representation significantly outperforms the standard graph representations by reducing computation costs and memory accesses for neighborhood aggregations by up to 84\% and YY\%, and increasing end-to-end training throughput by up to 1.9$\times$, while maintaining the original model accuracy.
}
\end{abstract}

\section{Introduction}
\label{sec:intro}
Graph neural networks (GNNs) have shown state-of-the-art performance across a number of tasks with graph structured data, such as social networks, molecule networks, and webpage graphs~\cite{GCN, GraphSAGE, DiffPool, GIN}. 
GNNs use a recursive neighborhood aggregation scheme --- in a GNN layer, each node aggregates its neighbors' activations from the previous GNN layer and uses the aggregated value to update its own activations.
The activations of the final GNN layer are used for prediction tasks, such as node classification, graph classification, or link prediction.

Due to the clustering nature of real-world graphs, different nodes in a graph may share a number of common neighbors.
For example, in webpage graphs, different websites under the same domain generally have a number of common links (i.e., neighbors).
As another example, in recommender systems, users in the same group may have interests in common items.

Existing GNNs define computation in each GNN layer with a GNN {\em computation graph} (GNN-graph). 
For each node $v$ in the input graph, the GNN-graph includes an individual tree structure to describe how to compute the node's activations by aggregating the previous-layer activations of its neighbors.
%Existing GNNs define computation in neighborhood aggregations with a GNN computation graph that includes an individual aggregation operator to aggregate each node's neighbors.
%%To perform neighborhood aggregations in a GNN layer, existing GNNs use a {\em standard} format for computation graphs with a number of individual aggregation operators, each of which aggregates the neighbors of a vertex in an input graph.
%Figure~\ref{fig:intro}a shows an example input graph. 
Figure~\ref{fig:intro}b shows the GNN-graph representation of the input graph in Figure~\ref{fig:intro}a;
for example, for node $A$, its neighbor's activations $h^{(k-1)}_B$, $h^{(k-1)}_C$ and $h^{(k-1)}_D$ from the layer $k-1$ are aggregated to compute new activations $h^{(k)}_A$ for the layer $k$ (see the top portion of Figure~\ref{fig:intro}b).
The new activations of the other nodes are computed similarly using the previous activations of their neighbors.
Notice that this results in redundant computation and data transfers. In this example, both $\{A,B\}$ and $\{C,D\}$ are aggregated twice.
In practice, the level of redundancy is much greater in real graphs with many more nodes and multiple layers.

%Figure~\ref{fig:intro}b shows an example standard computation graph.
%This approach performs neighborhood aggregations independently on each vertex, resulting in redundant computation and data transfers since aggregations on the shared subsets of vertices are performed multiple times (e.g., both $\{A, B\}$ and $\{C, D\}$ are aggregated twice in Figure~\ref{fig:intro}b).

\begin{figure*}
    \centering
    \includegraphics[scale=0.5]{figures/intro3.pdf}
    \caption{Comparison between a GNN-graph and an equivalent \xg.
    (a) Input graph; (b) 1-layer GNN-graph; (c) \xg that avoids redundant computation.
    The GNN-graph computes new activations $h^{(k)}_v$ by aggregating the previous-layer activations of its network neighbors.
    Because nodes in the input graph share common neighbors, the GNN-graph performs redundant computation (e.g., both $\{A, B\}$ and $\{C, D\}$ are aggregated twice). By identifying common computational patterns, the \xg avoids repeated computation.
    }
    \label{fig:intro}
\end{figure*}

In this paper, we propose a new GNN representation called {\em Hierarchically Aggregated computation Graphs} (\xgs).
%, which eliminate redundant computation and unnecessary data transfers in GNN computation graphs by hierarchically managing and reusing intermediate aggregation results. 
Figure~\ref{fig:intro}c shows one possible \xg for the input graph in Figure~\ref{fig:intro}a. %, which computes neighborhood aggregations hierarchically to reuse intermediate aggregations.
%Compared to standard computation graphs, \xgs 
\xgs are functionally equivalent to standard GNN-graphs (produce the same output), but hierarchically combine redundant computation and remove unnecessary data transfers.
%, while preserving the exact same output as the standard computation graphs.
Note that a \xg is agnostic to any particular GNN model and can be directly used to eliminate redundancy for arbitrary GNN computation graphs.
%a new graph representation for GNNs called {\em hierarchical aggregation graphs} (\xgs), which eliminate redundant computation and reduce unnecessary data accesses in GNNs by reusing intermediate aggregation results hierarchically.
%\xg includes a number of intermediate {\em aggregation vertices}, each of which represents the aggregation result of a specific subset of vertices and can be reused for the neighborhood aggregations of multiple vertices.

%Compared to the original graph representation that directly link each vertex to all neighbors, our HSG representation eliminates redundant aggregations on XXX and reduces memory accesses by reusing intermediate subset vertices.

For a GNN-graph, there exist numerous equivalent \xgs with different aggregation hierarchies and performance.
%Our goal is to find a \xg with optimized runtime performance while preserving original model accuracy. 
%To formalize the problem, w
We introduce a cost model to quantitatively evaluate the runtime performance of different \xgs and develop a novel \xg search algorithm to automatically find optimized \xgs.

Theoretically, we prove that: (1) for GNNs whose neighborhood aggregations require a specific ordering on a node's neighbors, the \xg search algorithm can find a globally optimal \xg under the cost model; and (2) for other GNNs, the algorithm provides a $(1-1/e)$-approximation of globally optimal \xgs under the cost model.
Empirically, the \xg search algorithm finds highly optimized \xgs for real-world graph datasets, reducing the number of aggregations in GNN-graphs by up to 6.3$\times$.
%For GNN models whose aggregations require a specific ordering on a vertex's neighbors, the greedy algorithm is able to find a globally optimal \xg under the cost model.
%For other GNN models, our algorithm has no optimality guarantee but empirically reduces computation costs of neighborhood aggregations by up to 6.7$\times$.

%Existing GNN implementations only achieve suboptimal runtime performance.
Existing deep learning frameworks such as TensorFlow and PyTorch train GNNs by translating GNN-graphs to sparse matrices and performing matrix operations.
Besides being less efficient than \xgs, this approach does not consider graph structures in GNNs and disables a number of critical system optimizations for graphs (see Section~\ref{sec:impl}).

Based on the above insights, we implemented \xg in a GNN framework we call \sys.
The key difference between \sys and existing frameworks is that \sys explicitly manages graph structures in GNNs and reduces GNN training to a number of graph processing operations. This allows \sys to directly benefit from system optimizations for graphs.

Our \xg abstraction maintains predictive performance of GNNs but leads to much faster training and inference. We evaluate the performance of \sys on five real-world graph datasets and along three dimensions: (a) end-to-end training performance; (b) number of aggregations; and (c) amount of data movement in GNN training.
%We evaluate the runtime performance of \sys with \xgs on five real-world graph datasets including social networks~\cite{GraphSAGE}, molecule networks~\cite{BZR, PPI}, and scientific collaboration datasets~\cite{COLLAB}.
Experiments show that \sys significantly outperforms state-of-the-art deep learning frameworks for GNN training, with end-to-end speedups ranging from 4.6$\times$ to 15.3$\times$.
%that supports training arbitrary GNNs on \xgs and significantly outperforms state-of-the-art deep learning frameworks on five real-world graph datasets, with speedups ranging from 4.6$\times$ to 15.3$\times$.
%The performance improvement over the baselines is twofold.
First, \sys enables a number of system optimizations for graphs by reducing GNN training to a sequence of graph processing operations, which increases the training throughput by 3.7-5.5$\times$. Second, \sys uses the \xg representation to eliminate redundant aggregations and data movement in GNN training, which further increases the training throughput by up to 2.8$\times$.
In addition, compared to GNN-graphs, \xgs reduce the number of aggregations and the amount of data transfers in GNN training by up to 6.3$\times$ and $5.6\times$, respectively.
%We have implemented a fast GNN framework that supports training arbitrary GNN models on \xgs. 
%\ZJ{talk about performance comparison between our framework and the baselines.}
%significantly outperforms existing GNN frameworks with speedups ranging from 6.2$\times$ to 8.7$\times$.

%Compared to standard computation graphs, \xgs improves the training throughput of GNN models by up to 2.3$\times$. In addition, \xgs can reduce the computation costs and data transfers for neighborhood aggregations by up to 6.7$\times$ and 2.2$\times$, respectively.
%\ZJ{\xgs allow more efficient graph implementation. }

To summarize, our contributions are: 
\begin{itemize}
    \setlength\itemsep{0em}
    \item We propose \xg, a new GNN graph representation to eliminate redundant computation and data transfers in GNN training.
    \item We define a cost model to quantitatively evaluate the runtime performance of different \xgs and develop a \xg search algorithm to automatically find optimized \xgs.
    We prove that the \xg search algorithm provides at least a $(1-1/e)$-approximation of globally optimal \xgs under the cost model.
    \item We show that \xgs significantly outperform GNN-graphs by increasing GNN training throughput by up to 2.8$\times$ and reducing the aggregations and data transfers in GNN training by up to 6.3$\times$ and 5.6$\times$, respectively.
\end{itemize}

\section{Graph Neural Network Abstraction}
\begin{algorithm}[t]
\caption{An abstraction for GNNs. $\mathcal{V}$ is the set of nodes in an input graph, and $\mathcal{N}(v)$ denotes the set of neighbors for node $v$.}
\label{alg1}
\begin{algorithmic}[1]
\State $h^{(0)}_v = x_v, \forall v \in \mathcal{V}$
\For {$k= 1 \textrm{ to } K$}
\For {$v \in {\mathcal{V}}$}
\State $a^{(k)}_v \leftarrow \Call{Aggregate}{\{h^{(k-1)}_u | u \in \mathcal{N}(v)\}}$
\State $h^{(k)}_v \leftarrow \Call{Update}{a^{(k)}_v, h^{(k-1)}_v}$
\EndFor
\EndFor
\State
\State {\bf Goal:} minimize $\mathcal{L}(\{h^{(K)}_v | v \in \mathcal{V}\})$
\end{algorithmic}
\end{algorithm}
A GNN takes an input graph and node features as inputs and iteratively learns representations for individual nodes over the entire graph through a number of GNN layers.
%Each GNN layer consists of two steps.
Algorithm~\ref{alg1} shows an abstraction for GNNs: $h^{(k)}_v$ is the learned activations of node $v$ at layer $k$, and we initialize $h^{(0)}_v$ with input node features $x_v$.
At the $k$-th layer, $a^{(k)}_v$ denotes the aggregated activations of $v$'s neighbors, which is combined with $h^{(k-1)}_v$ to compute an updated activation $h^{(k)}_v$.
The learned node activations of the final layer (i.e., $h^{(K)}_v$) are used for predictions, and a GNN model generally minimizes a loss function $\mathcal{L}$ that takes the final node activations as inputs (line 9).

\begin{table*}[ht]
\caption{Existing GNNs described in our \xg abstraction. GraphSAGE-P and GraphSAGE-LSTM are the pooling and LSTM variants of GraphSAGE, respectively. $\sigma$ and $\er{max}$ indicate element-wise non-linear activation and max functions.
For sequential \textproc{Aggregate}, $v_i$ denotes the $i$-th in-neighbor of node $v$.
}
\label{tab:gnns}
\begin{threeparttable}
\resizebox{\textwidth}{!}{
\begin{tabular}{lll}
\hline
{\bf GNN} & {\bf $\textproc{Aggregate}(\{h^{(k-1)}_u | u \in \mathcal{N}(v)\})$} & {\bf $\textproc{Update}(a^{(k)}_v, h^{(k-1)}_v)$}\\
\hline
\multicolumn{3}{c}{Set \textproc{Aggregate}} \\
\hline
GCN~\cite{GCN} & $a^{(k)}_v = \sum_{u\in\mathcal{N}(v)}{h^{(k-1)}_u} $ & $h^{(k)}_v = \sigma(W^{(k)} \cdot \frac{a^{(k)}_v + h^{(k-1)}_v}{|\mathcal{N}(v)| + 1})$ \\
GIN~\cite{GIN} & $a^{(k)}_v = \sum_{u\in\mathcal{N}(v)}{h^{(k-1)}_u} $ & $h^{(k)}_v = \sigma\big(W \cdot \Big((1 + \epsilon^{(k)})h^{(k-1)}_v + a^{(k)}_v\big) \Big)$ \\
GraphSAGE-P~\cite{GraphSAGE} & $a^{(k)}_v = {\er{max}}_{u\in\mathcal{N}(v)}\{\sigma(W^{(k)}_1 \cdot h^{(k-1)}_u)\}$ & $h^{(k)}_v = \sigma\big(W^{(k)}_2 \cdot (a^{(k)}, h^{(k-1)}_v)\big)$ \\
%GraphSAGE~\cite{GraphSAGE} & $a^{(k)}_v = \frac{1}{|\mathcal{N}(v)|}\sum_{u\in\mathcal{N}(v)}{h^{(k-1)}_u}$ & $h^{(k)}_v = \sigma\big(W^{(k)} \cdot (a^{(k)}_v, h^{(k-1)}_v)\big)$\\
%\hline
%\multicolumn{3}{c}{Idempotent \textproc{Aggregate}} \\
%\hline
%GCN-P~\cite{GCN} & $a^{(k)}_v = {\er{max}}_{u\in\mathcal{N}(v)}\{h^{(k-1)}_u\} $ & $h^{(k)}_v = \sigma\big(W^{(k)} \cdot (a^{(k)}_v | h^{(k-1)}_v)\big)$ \\
\hline
\multicolumn{3}{c}{Sequential \textproc{Aggregate}} \\
\hline
GraphSAGE-LSTM~\cite{GraphSAGE} & $a^{(k)}_v = \er{LSTM}(h^{(k-1)}_{v_1},...,h^{(k-1)}_{v_\mathcal{N}})$ & $h^{(k)}_v = \sigma\big(W^{(k)} \cdot (a^{(k)}_v, h^{(k-1)}_v)\big)$\\
$N$-ary Tree-LSTM~\cite{TreeLSTM} & $a^{(k)}_v = \er{Tree-LSTM-Agg}(h^{(k-1)}_{v_1},...,h^{(k-1)}_{v_\mathcal{N}})$ & $h^{(k)}_v = \er{Tree-LSTM-Update}(a^{(k)}_v, h^{(k-1)}_v)$\\
\hline
\end{tabular}
}
\end{threeparttable}
\end{table*}



\section{Hierarchically Aggregated Computation Graphs (\xgs)}
\label{subsec:graph}

Existing GNN models use a GNN {\em computation graph} (GNN-graph) to describe the computation in each GNN layer, as shown in Figure~\ref{fig:intro}b.
For each node $v$ in the input graph, the GNN-graph includes an individual tree structure to define how to compute the activations $h_v^{(k)}$ of node $v$ by aggregating the previous-layer activations of $v$'s neighbors (i.e., $h^{(k-1)}_u, u \in \mathcal{N}(v)$).
GNN-graphs are efficient at capturing direct neighborhood relations between nodes but include redundant computation and data transfers since aggregations on shared neighbors are performed multiple times.

\hide{
$\m{G}=(\langle\m{V}_S, \m{V}_A\rangle, \m{E})$ to describe neighborhood aggregations in a GNN layer.
Each node $v \in \m{V}_S$ denotes $v$'s activations at the previous layer (i.e., $h_v^{k-1)}$, and a node $u \in \m{V}_A$ corresponds to the aggregated activations of $u$'s neighbors (i.e., $a_u^{(k)}$ in Algorithm~\ref{alg1}). There is an edge from $v \in \m{V}_S$ to $u \in \m{V}_A$ if $v$ and $u$ are neighbors in the input graph.
Recall Figure~\ref{fig:intro}b, which shows an example standard computation graph.
This approach is efficient at capturing direct neighborhood relations between nodes but includes redundant computation and data transfers since aggregations on shared neighbors are performed multiple times.
}

%Existing GNN models use an {\em standard} format of computation graphs to describe neighborhood aggregations in a GNN layer (see Figure~\ref{fig:intro}b).
%An standard computation graph includes an independent aggregator for each node to aggregate its neighbors in the input graph.
%This approach is efficient at capturing pair-wise relations between nodes but does not consider common sets of neighbors shared among multiple nodes.
%Training GNN models directly on standard computation graphs results in redundant computation since aggregations on the shared neighbors are performed multiple times.
%%This approach does not consider reusing the intermediate results of aggregating commonly used subset of nodes and results in redundant computation. 

%GNNs learn graph topology by aggregating each node's neighbors in each layer (i.e., \textproc{Aggregate} in Algorithm~\ref{alg1}).
We propose {\em Hierarchically Aggregated computation Graphs} (\xgs) for GNNs, which eliminate redundancy in GNN-graphs by hierarchically managing and reusing intermediate aggregation results.
%Compared to the original graph representation, \xg reduces both computation and data transfer costs in GNNs.
%Compared to an standard computation graph, a
Compared to a GNN-graph, a \xg includes a new set of {\em aggregation} nodes, each of which represents the intermediate aggregations results for a subset of nodes (i.e., aggregation on a subset of $h^{(k-1)}_v$).
Similar to edges in GNN-graphs, an edge $(u, v)$ in a \xg denotes an aggregation relation --- computing $v$'s activations requires aggregating $u$'s activations.

%A \xg $\mw{G} = (\langle\m{V}_S, \m{V}_A, \m{V}_I\rangle, \mw{E})$ includes a third set of nodes $\m{V}_I$, each of which represents the intermediate aggregation results for a subset of nodes (i.e., aggregation on a subset of $h^{(k-1)}_v$).
%A \xg $\mw{G} = (\langle\m{V}_S, \m{V}_A, \m{V}_I\rangle, \mw{E})$ is a directed acyclic graph with three types of nodes 
%(i.e., $\mathcal{\widehat{V} = \widehat{V}_S + \widehat{V}_D + \widehat{V}_I}$). 
%First, for each node $v$ in a given training graph, $\mathcal{\widehat{V}}_S$ includes a {\em source node} corresponding to $v$'s activations at the previous layer (i.e, $h^{(k-1)}_v$). 
%Second, for each node $v$ in the training graph, $\mathcal{\widehat{V}}_E$ has a {\em destination node} corresponding to the aggregated activations of $v$'s neighbors (i.e., $a^{(k)}_v$).
%Finally, $\mathcal{\widehat{V}}_I$ contains a number of intermediate {\em subset node}, each of which is the aggregated activations for a subset of nodes (i.e., aggregation on a subset of $h^{(k-1)}_v$).

%Edges in $\mathcal{H}$ denotes aggregation --- all in-edges of each node are aggregated and used as the node's representation.
%First, $\er{depth}(v)$ is the length of the longest path from a source node to $v$, which describes the depth of a node in the hierarchy.
%$$
%\er{depth}(v) = \begin{cases}
%0 & v \in \widehat{\mathcal{V}}_S \\
%\max_{(u, v) \in \widehat{\mathcal{E}}} \{\er{depth}(u) + 1\} & v \in %\mathcal{\widehat{V}}_I \cup \mathcal{\widehat{V}}_D
%\end{cases}
%$$

Our \xg abstraction is general and applicable to many existing GNN models.
Table~\ref{tab:gnns} shows how to use our abstraction to define existing GNNs, which can be further divided into two categories based on their \textproc{Aggregate} functions.

\begin{itemize}
\setlength\itemsep{0em}
\item {\bf Set \textproc{Aggregate}}. Most GNNs assume the neighbors of a node have {\em no ordering}, and the aggregations are {\em associative} and {\em commutative} operations that are invariant to the order in which the aggregations are performed. Examples include GCN and GIN with summation aggregations and GraphSAGE-P with element-wise pooling aggregations (Table~\ref{tab:gnns}).
Note that set aggregations in GNNs are designed to be order invariant and thus can be performed in a hierarchical fashion as we do in \xgs.
\hide{
%Note that aggregation functions in GNNs are designed to be order invariant which is exactly the property we need for \xgs  to work. 
%Essentially, set aggregations are associative and commutative and aggregations can thus be performed in a hierarchical fashion as we do in \xgs.
}
%\ZJ{We define an \textproc{Aggregate} function to be {\em unordered} if the aggregation results are permutation invariant.}
%and requires each element to be aggregated exactly once.
%\item {\bf Idempotent \textproc{Aggregate}}. A second class of GNNs assume {\em no ordering} on the neighbors of a node and allows each neighbor to be aggregated {\em multiple} times.
%For example, both GCN-P and GraphSAGE-P uses an element-wise max-pooling aggregator, which allows aggregating some neighbors multiple times and preserves the same outputs. 
%We call this an {\em idempotent} \textproc{Aggregate}.

\item {\bf Sequential \textproc{Aggregate}}. Another class of GNNs require a specific ordering of a node's neighbors and the aggregations are not commutative. 
Examples include $N$-ary Tree-LSTM~\cite{TreeLSTM} and the LSTM variant of GraphSAGE~\cite{GraphSAGE}.
However, \xgs can be applied in the case of sequential aggregations as well. 
Rather than identifying common subsets of neighbors, we identify the common prefixes of the sequence of aggregated nodes, which can then be shared among nodes to reduce computation.
%For GNNs with a {\em sequential} \textproc{Aggregate}, the neighbor set $\mathcal{N}(v)$ is an ordered list.
%describing the order in which \textproc{Aggregate} should be performed on the neighbors.
\end{itemize}

%\ZJ{say why we need to define two types of aggregators}

{\bf Formal definition of \xgs.}
We use $\m{V}$ to denote the nodes in the input graph and use $\m{V}_A$ to denote the aggregation nodes added in a \xg.
A GNN-graph is a special case in the \xg representation with no intermediate aggregation nodes (i.e., $\m{V}_A = \emptyset$). 
We further define two additional functions for each node:

First, $\er{aggr}(v)$ is the aggregation results of node $v$:
$$
\er{aggr}(v) = \textproc{Aggregate}(\{\er{aggr}(u) | u \in \mw{N}_v\})
$$
where $\mw{N}_v$ denotes the in-neighbors of node $v$ in a \xg.
Note that $\er{aggr}(\cdot)$ is recursively defined, and there exist a sequential ordering to evaluate $\er{aggr}(v)$ for all nodes since each \xg is acyclic.

\hide{
First, $\er{aggr}(v)$ is the result of aggregating $v$'s in-neighbors with an \textproc{Aggregate} function.
For a source node $v$ with no in-neighbors, $\er{aggr}(v)$ is $v$'s activations in the previous layer:
$$
\er{aggr}(v) = \begin{cases}
h^{(k-1)}_v & v \in \m{V}_S \\
\textproc{Aggregate}(\{\er{aggr}(u) | (u, v) \in \mathcal{\widehat{E}}\}) & v \in \m{V}_A \cup \m{V}_I
\end{cases}
$$
Note that $\er{aggr}(\cdot)$ is recursively defined, and there exist a sequential ordering to evaluate $\er{aggr}(v)$ for all nodes since $\widehat{\mathcal{G}}$ is acyclic.
}

Second, we use $\er{cover}(v)$ to describe how to compute $\er{aggr}(v)$ by using the input activations $h^{(k-1)}_u$ from the previous layer.
%we define $\er{cover}(v)$ as the set of nodes whose activations are aggregated to compute $\er{aggr}(v)$.
\begin{equation}
\er{aggr}(v) = \textproc{Aggregate}(\{h^{(k-1)}_u | u \in \er{cover}(v)\}
\end{equation}
$\er{cover}(v)$ defines the coverage of node $v$ in a \xg. For the \xg example in Figure~\ref{sec:intro}c, $\er{cover}(A) =  \{B, C, D\}$ because $h^{(k-1)}_A$, $h^{(k-1)}_B$, and $h^{(k-1)}_C$ are used as inputs to compute the aggregated results of node $A$.

For a set \textproc{Aggregate}, $\er{cover}(\cdot)$ is an unordered set: % and can be calculated with the following equation.
\begin{equation}
\label{eqn1}
\er{cover}(v) = \{w | \exists u \in \mw{N}_v: w \in \er{cover}(u)\}
\end{equation}

For a sequential \textproc{Aggregate}, $\er{cover}(\cdot)$ is an ordered list:
\begin{equation}
\label{eqn2}
\er{cover}(v) = \big(\er{cover}(u_1), ..., \er{cover}(u_m)\big)
\end{equation}
where $u_1, ..., u_m$ are the ordered in-neighbors of $v$.
%Theorem~\ref{thm1} shows how to compute $\er{cover}(v)$ for different types of aggregators. We prove the theorem in Appendix. 

%\begin{theorem}
%\label{thm1}
%For a source node $v$, $\er{cover}(v) = \{v\}$ by definition.
%\begin{eqnarray*}
%\er{cover}_{\rm ord}(v) & = &\{ w | \exists! (u, v)\in\mathcal{\widehat{E}}: w \in \er{cover}_{\rm ord}(u) \} \\
%\er{cover}_{\rm ide}(v) & = &\{ w | \exists (u, v)\in\mathcal{\widehat{E}}: w \in \er{cover}_{\rm ide}(u) \} \\
%\er{cover}_{\rm seq}(v) & = &( \er{cover}_{\rm seq}(u_1), ... , \er{cover}_{\rm seq}(u_m))\\
%\end{eqnarray*}
%where $(u_1, v), ..., (u_m, v)$ are ordered in-edges of $v$.
%\end{theorem}

%\begin{eqnarray*}
%& \er{cover}({\textproc{Aggregate}}, v) \\
%& = \begin{cases}
%\{ w | \exists! (u, v)\in\mathcal{\widehat{E}}: w \in \er{cover}(\textproc{Aggregate}, u) \} & \textrm{\textproc{Aggregate} is standard} \\
%\{ w | \exists (u, v)\in\mathcal{\widehat{E}}: w \in \er{cover}(\textproc{Aggregate}, u) \} & \textrm{\textproc{Aggregate} is idempotent} \\
%( \er{cover}(\textproc{Aggregate}, u_1), ... , \er{cover}(\textproc{Aggregate}, u_m)) & \textrm{\textproc{Aggregate} is sequential}
%\end{cases}
%\end{eqnarray*}

\subsection{GNNs with \xgs}
\begin{algorithm}[t]
\caption{A GNN abstraction with \xgs. $\widehat{a}_v$ denotes the result of $\er{aggr}(v)$ at a GNN layer. We exclude layer index superscripts in $\widehat{a}_v$ to denote that $\widehat{a}_v$ does not need to be memorized for back propagation,
and its memory can be reused across all layers.}
\label{alg2}
\begin{algorithmic}[1]
\State $h^{(0)}_v = x_v, \forall v \in \m{V}$
\For {$k= 1 \textrm{ to } K$}
\For {$v \in {\m{V}}$}
\State $\widehat{a}_v \leftarrow h^{(k-1)}_v$
\EndFor
\For {$v \in \m{V}_A$}
\State $\widehat{a}_v \leftarrow \Call{Aggregate}{\{\widehat{a}_u | u \in \mw{N}_v\}}$
\EndFor
\For {$v \in {\m{V}}$}
\State $a^{(k)}_v \leftarrow \Call{Aggregate}{\{\widehat{a}_u | u \in \mw{N}_v\}}$
\State $h^{(k)}_v \leftarrow \Call{Update}{a^{(k)}_v, h^{(k-1)}_v}$
\EndFor
\EndFor
\end{algorithmic}
\end{algorithm}

Existing GNNs are defined with GNN-graphs as shown in Algorithm~\ref{alg1}.
We extend the GNN abstraction in Algorithm~\ref{alg2} to make it also applicable to \xgs.
%Algorithm~\ref{alg2} shows the extended GNN abstraction for \xgs.
The extension does not require any modification to a GNN model, and the only difference is how to compute neighborhood aggregations (i.e., $a^{(k)}_v$) in each GNN layer.
In Algorithm~\ref{alg2}, we first compute the results of intermediate aggregation nodes and save the results in $\widehat{a}_v$ (line 6-8).
We then compute the neighborhood aggregations (i.e., $a^{(k)}_v$) for nodes in the input graph by opportunistically reusing the intermediate aggregation results $\widehat{a}_v$, which eliminates redundant computation and data transfers.

Note that, although Algorithm~\ref{alg2} includes new intermediate variables $\widehat{a}_v$, the memory overhead for storing $\widehat{a}_v$ is negligible since $\widehat{a}_v$ is not used for back propagation and can be saved in a constant memory across all layers.

We define a GNN-graph $\m{G}$ and a \xg $\mw{G}$ to be {\em equivalent} for a GNN model if (1) the GNN model outputs the same activations (i.e., $h^{(k)}_v$) at each layer, and (2) the GNN model computes the same gradients for all trainable parameters in back propagation. 
We can use equivalent graphs interchangeably for both inference and training, since equivalent graphs produce the same outputs and gradients by definition.
Theorem~\ref{thm2} provides a necessary and sufficient condition on graph equivalence. We prove the theorem in Appendix.
\begin{theorem}
\label{thm2}
A GNN-graph $\m{G}$ and a \xg $\mw{G}$ are equivalent if and only if $\mathcal{N}(v) = \er{cover}(v)$ for all $v \in \m{V}$, where $\mathcal{N}(v)$ is $v$'s neighbors in the input graph and $\er{cover}(\cdot)$ is defined in Equation~\ref{eqn1} and~\ref{eqn2}.
\end{theorem}

%\textbf{Equality between standard graphs and hierarchical aggregation graphs.}

Equivalent graphs achieve the same model accuracy but have different runtime performance. 
Theorem~\ref{thm2} provides an efficient way to check equivalence between GNN-graphs and \xgs, and can be used as an oracle to search for optimized \xgs for any GNN-graph.

\section{\xg Search Algorithm}
For an arbitrary GNN model and GNN-graph, our goal is to find an equivalent \xg with optimized runtime performance.
We define a cost model to quantitatively evaluate the runtime performance of arbitrary \xgs (Section~\ref{subsec:cost}) and introduce a \xg search algorithm that automatically finds an optimized \xg (Section~\ref{subsec:greedy}).

Theoretically, we show that:
\begin{itemize}
\item For GNNs with sequential \textproc{Aggregate}, the \xg search algorithm can find {\em globally optimal} \xgs under the cost model.
\item For GNNs with set \textproc{Aggregate}, finding an optimal \xg is NP-hard by a reduction from the NP-hard {\em maximum coverage problem} (see Appendix for the proof). The \xg search algorithm finds a {\em $(1-1/e)$-approximation} of globally optimal \xgs under the cost model.
\end{itemize}
%Our \xg search algorithm can empirically find highly optimized \xgs for GNNs with set \textproc{Aggregate}.
%By using the cost model, The greedy algorithm can find globally optimal \xgs for GNNs with sequential \textproc{Aggregate} and highly efficient \xgs for GNNs with unordered \textproc{Aggregate}.
\subsection{Cost Model}
\label{subsec:cost}
Our cost model assigns a cost to one epoch of GNN training on the \xg.

The computation cost of a GNN model includes aggregating the neighbors of each node by calling \textproc{Aggregate} and updating the activations of each node via \textproc{Update}, as shown in Algorithm~\ref{alg2}. 
For a GNN model $\m{M}$, we assume the cost of performing \textproc{Aggregate} on two elements is $\alpha_{\m{M}}$, and the cost of computing an \textproc{Update} is $\beta_{\m{M}}$.
%For a GNN model $\mathcal{M}$, we assume the cost of performing \textproc{Aggregate} on two elements is $\alpha$, and the cost of performing an \textproc{Update} is $\beta$.
In Algorithm~\ref{alg2}, computing $\widehat{a}_v$ with $|\mathcal{\widehat{N}}_v|$ neighbors requires performing $(|\mathcal{\widehat{N}}_v|-1)$ binary aggregations, whose cost is $\alpha_{\m{M}}\times(|\mathcal{\widehat{N}}_v|-1)$.
Therefore, the total computation cost of training a GNN model $\mathcal{M}$ on a \xg $\mw{G}$ is
\begin{eqnarray*}
\er{cost}(\mathcal{M}, \mathcal{\widehat{G}}) & = &\sum_{v \in \m{V} \cup \m{V}_A} \alpha_{\m{M}} (|\mathcal{\widehat{N}}_v| - 1) + \sum_{v \in \m{V}} \beta_{\m{M}} \\
& = & \alpha_{\m{M}} \big(|\mw{E}| - |\m{V}| - |\m{V}_A|\big) + \beta_{\m{M}} |\m{V} |\\
& = & \alpha_{\mathcal{M}}\big(|\mw{E}| - |\m{V}_A|\big) + (\beta_{\mathcal{M}} - \alpha_{\mathcal{M}}) |\m{V}|
\end{eqnarray*}
%where $\alpha_{\mathcal{M}} = \sum_{l \in \mathcal{M}}\alpha_l$ and $\beta_{\mathcal{M}} = \sum_{l \in \mathcal{M}} \beta_l$ are the overall aggregation and update costs of $\mathcal{M}$, respectively.
%The last equation is because $\mathcal{\widehat{G}}$ is equivalent to the ordinary graph $\mathcal{G}$ only if $\mathcal{\widehat{V}}_D = \mathcal{V}$.
Since $|\mathcal{V}|$ is determined by the input graph, our goal is to find a \xg minimizing $\big(|\mathcal{\widehat{E}}| -  |\mathcal{\widehat{V}}_A| \big)$.

\subsection{Search Algorithm}
\label{subsec:greedy}
\begin{algorithm}[t]
%\footnotesize
\caption{A \xg search algorithm to automatically find an equivalent \xg for a GNN-graph with optimized runtime performance.
\textproc{Redundancy}($v_1, v_2, \mw{E}$) calculates the number of nodes aggregating both $v_1$ and $v_2$.
$\m{V}_A$ is the set of aggregation nodes in a \xg. 
Recall that $\er{cover}(u)$ is an ordered list for sequential \textproc{Aggregate} (see Equation~\ref{eqn2}).}
\label{alg3}
\begin{algorithmic}[1]
\State {\bf Input: } A GNN-graph $\mathcal{G}$ and a GNN model $\m{M}$.
%the maximum number of aggregation nodes $\er{capacity}$, the maximum depth of aggregation nodes $\er{depth}$.
\State {\bf Output: } An equivalent \xg 
\State 
\Function{Redundancy}{$v_1, v_2, \mw{E}$}
\If {$\m{M}$ has a set \textproc{Aggregate}}
\State $\m{R} = \{ u | (v_1, u) \in \mw{E} \wedge (v_2, u) \in \mw{E}\}$
\Else
\State $\m{R} = \{ u | v_1 = \er{cover}(u)[1] \wedge v_2 = \er{cover}(u)[2]\}$
\EndIf
\State \textbf{return} $|\mathcal{R}|$
\EndFunction
%\Function{Depth}{$v$}
%\State \textbf{return} $\max\{\Call{Depth}{u} + 1 | (u, v) \in \mw{E}\}$
%\EndFunction
\State
\State $\m{V}_A \leftarrow \emptyset, \mw{E} \leftarrow \mathcal{E}$
\While {$|\m{V}_A| < \er{capacity}$}
\State $(v_1, v_2) = \argmax_{v_1, v_2}$ \Call{Redundancy}{$v_1, v_2, \mw{E}$}
\If {$\Call{Redundancy}{v_1, v_2, \mw{E}} > 1$}
\State $\m{V}_I \leftarrow \m{V}_I + \{w\}$ \Comment{where $w$ is a new node}
\State $\mw{E} \leftarrow \mw{E} + (v_1, w) + (v_2, w)$
\For {$u \in \m{V}$}
\If {$(v_1, u) \in \mw{E} \wedge (v_2, u) \in \mw{E}$}
\State $\mw{E} \leftarrow \mw{E} - (v_1, u) - (v_2, u) + (w, u)$
\EndIf
\EndFor
\EndIf
\EndWhile
\State {\bf return } $(\m{V}_A, \mw{E})$
\end{algorithmic}
\end{algorithm}

%\begin{figure}
%\centering
%\subfloat[Input graph.]{
%\includegraphics[scale=0.3]{figures/graph_example.pdf}
%}
%\\
%\subfloat[Initial \xg with 9 binary aggregations.]{
%\includegraphics[scale=0.3]{figures/step0.pdf}
%}
%\\
%\vspace{-2mm}
%\subfloat[Updated \xg with 7 binary aggregations.]{
%\includegraphics[scale=0.3]{figures/step1.pdf}
%}
%\\
%\vspace{-2mm}
%\subfloat[Updated \xg with 6 binary aggregations.]{
%\includegraphics[scale=0.3]{figures/step2.pdf}
%}
%\\
%\vspace{-2mm}
%\subfloat[Final \xg with 5 binary aggregations.]{
%\includegraphics[scale=0.3]{figures/step3.pdf}
%}
%\vspace{-2mm}
%\caption{Iteratively constructing an \xg with the graph search algorithm. nodes with a red box indicate the chosen nodes at each iteration.}
%\label{fig:greedy}
%\end{figure}
%\ZJ{Say why it is hard to find an optimal solution}
We propose a \xg search algorithm that finds a globally optimal \xg for GNNs with sequential \textproc{Aggregate} and a $(1-1/e)$-approximation of globally optimal \xgs for GNNs with set \textproc{Aggregate}.
In addition to an input GNN-graph and a GNN model, the search algorithm also takes a parameter {\em capacity}, defining an upper limit on the number of intermediate aggregation nodes (i.e., $|\m{V}_A|$) in the \xg.
%that specify the maximum capacity and depth of all aggregation nodes in the $\m{V}_A$, respectively.
%The depth of a node $v$ is the length of the longest path from a node to $v$, which describes the latency to compute node $v$ since all nodes along the longest path must be computed sequentially.
%The capacity is an upper limit on $|\m{V}_A|$.

Algorithm~\ref{alg3} shows the pseudocode of the \xg search algorithm.
We start with an input GNN-graph, which is a special case of \xgs, and iteratively insert aggregation nodes into the current \xg to merge highly redundant aggregations and remove unnecessary computation and data transfers.

In each iteration, we find a binary aggregation with the highest redundancy and insert a new aggregation node $w$ in $\m{V}_A$ to represent the binary aggregation results (line 15-18).
%a pair of nodes $(v_1, v_2)$ with the highest redundancy and introduces a new node $w$ to represent the aggregation of $v_1$ and $v_2$.
All nodes containing this binary aggregation can directly use the output of $w$ without recomputing the aggregation (line 19-23).
The \xg search algorithm iteratively reduces the computation costs of the \xg by eliminating the most redundant binary aggregation in each iteration.
%All nodes that originally includes both $v_1$ and $v_2$ as in-neighbors now contains $w$ as an input. 
%This eliminates the redundant computation for aggregating $v_1$ and $v_2$ multiple times for different nodes.
%Figure~\ref{fig:greedy} demonstrates how the graph search algorithm iteratively generates an \xg for the ordinary graph in Figure~\ref{}.

For any GNN model with a sequential \textproc{Aggregate}, Theorem~\ref{thm3} shows that the \xg search algorithm is able to find an equivalent \xg with globally optimal computation cost. We prove the theorem in Appendix.

\begin{theorem}
\label{thm3}
For any GNN-graph $\m{G}$ and any GNN model $\m{M}$ with a sequential \textproc{Aggregate}, Algorithm~\ref{alg3} returns an equivalent \xg with globally minimized cost as long as 
$\er{capacity}\geq |\m{E}|$, where $|\m{E}|$ is the number of edges in $\m{G}$.
%For any \xg $\mw{G}$ that is equivalent to $\mathcal{G}$, $\er{cost}(\mw{G}_0) \leq \er{cost}(\mw{G}_0)$.
\end{theorem}

%We prove the correctness of Theorem~\ref{thm3} and the graph search algorithm in Appendix. 
%Theorem~\ref{thm3} shows that the graph search algorithm can find optimal \xgs for GNN models with sequential \textproc{Aggregate}.

For GNN models with set \textproc{Aggregate}, Theorem~\ref{thm4} shows that the \xg search algorithm can find an equivalent \xg that is within a $(1-1/e)$-approximation of the globally optimal \xgs. We prove the theorem in Appendix.
\begin{theorem}
\label{thm4}
For any GNN-graph $\m{G}$ and any GNN model $\m{M}$ with a set \textproc{Aggregate}, Algorithm~\ref{alg3} gives a $(1-1/e)$-approximation of globally optimal \xgs under the cost model. More specifically, assuming $\mw{G}$ is the \xg returned by Algorithm~\ref{alg3} and $\mw{G}_o$ is a globally optimal \xg under the $\er{capacity}$ constraint, we have
$$
\er{cost}(\m{M}, \mw{G}) \leq \frac{1}{e} \er{cost}(\m{M}, \m{G}) + \frac{e-1}{e} \er{cost}(\m{M}, \mw{G}_o)
$$
\end{theorem}
Empirically, the \xg search algorithm finds highly optimized \xgs for real-world graph datasets, reducing the number of aggregations by up to 6.3$\times$.
%{\bf Time Complexity.} 

%\begin{table}
%\caption{Time complexity of Algorithm~\ref{alg3}. |V| and |E| denote the number of nodes and edges in the input ordinary graph, respectively.}
%\begin{tabular}{|l|l|}
%\hline
%{\bf Step} & {\bf Time Complexity} \\
%\hline
%Initialize the heap structure & $O(|\mathcal{V}|^2)$ \\
%\hline
%Query the binary aggregations & \multirow{2}{*}{$O(\er{capacity} \times \log|\mathcal{V}|)$} \\
%with highest redundancy & \\
%\hline
%Update the heap structure & $O(|\mathcal{E}| \log|\mathcal{V}|)$\\
%\hline
%\hline
%Overall & $O(|\mathcal{V}|^2 + |\mathcal{E}| \log|\mathcal{V}|)$\\
%\hline
%\end{tabular}
%\end{table}
{\bf Time complexity.} Finding the binary aggregation with the highest redundancy in each iteration could be computationally very expensive, since a brute-force approach requires enumerating all node pairs.
We use a {\em heap} to maintain the redundancy score of each potential node pair and only update the heap when we add and remove edges in $\mw{E}$.
%Table~\ref{tab:} shows the time complexity of the graph search algorithm.
Since the depth of the heap is at most $O(\log(|\m{V}_S| + |\m{V}_A|))$~\footnote{This is because there can be at most $(|\m{V}_S| + |\m{V}_A|)^2$ node pairs.}, querying the most redundant binary aggregation and modifying $\mw{E}$ each takes $O(\log(|\m{V}_S| + |\m{V}_A|))$ time. 
%The total number of queries to the heap is $\er{capacity}$, since each query results in one node added into $\m{V}_I$.
%Meanwhile, the total number of updates to the heap is $O(|\mw{E}|)$.
%Therefore, the overall time complexity of the graph search algorithm is $O((\er{capacity} + |\mw{E}|)\times \log(|\m{V}_S| + |\m{V}_A|)$.
%The greedy algorithm achieves high efficiency on real-world graphs. 
For all the graph datasets used in our experiments, the \xg search algorithm takes at most 15 minutes to finish on a commodity Intel CPU.

In addition to reducing computation costs, the \xgs discovered by the \xg search algorithm have two other advantages.

{\bf Fast GPU implementation.} Most real-world graphs have non-uniform edge distributions, leading to unbalanced computation workload among different nodes.
Previous work~\cite{NGra, Lux} has proposed different strategies to explicitly balance workload distributions among nodes at the cost of synchronization overhead among GPU threads.
In contrast, our \xg search algorithm produces \xgs whose aggregation nodes (i.e., $\m{V}_A$) have uniform edge distributions (each has exactly two in-edges).
This eliminates any synchronization overheads to balance workload among aggregation nodes and results in faster GPU implementations.

{\bf High reusability.} For a given GNN-graph, the \xg discovered by the search algorithm only depends on the capacity and aggregation type (set or sequential \textproc{Aggregate}) and is agnostic to any particular GNN models.
This allows us to only run the search algorithm once for each aggregation type, and any GNN models can directly reuse the generated \xgs without any additional analysis on the graph.

\section{\xg Implementation}
\label{sec:impl}
%\begin{figure}
%    \centering
%    \includegraphics[scale=0.33]{figures/impl.pdf}
%    \caption{Performance comparison among TensorFlow, DGL with PyTorch backend, and our framework. We measure the per-epoch run-time to train a 2-layer GCN model on the Pubmed dataset~\cite{GCN} on a NVIDIA Tesla V100 GPU.}
%    \label{fig:compare_impl}
%\end{figure}
Existing deep learning frameworks such as TensorFlow~\cite{Tensorflow}, PyTorch~\footnote{https://pytorch.org/} and MXNet~\cite{MXNet} are designed for spatial data structures (e.g., images and text), and have limited supports for irregular data structures such as graphs.
As a result, GNN implementations in existing frameworks translate graph structures to sparse adjacent matrices and use matrix operations to perform GNN training. 
%GNN models implemented in existing deep learning frameworks such as TensorFlow~\cite{Tensorflow}, PyTorch~\footnote{https://pytorch.org/}, and DGL~\footnote{https://dgl.ai} translate graph structures to sparse matrices and use sparse matrix operations to perform GNN operations.
This approach disables a number of critical system optimizations for graphs, such as efficient load balancing among different nodes and cache optimizations to accelerate neighborhood aggregations~\cite{Lux}, resulting in suboptimal runtime performance.

Based on the above insights we implemented our \xg abstraction in \sys, a new deep learning framework for fast GNN training.
The key difference between \sys and existing deep learning frameworks is that \sys explicitly manages graph structures in GNNs and reduces GNN training to a number of graph processing operations, such as node gather/scatter operations.

The aggregations in each GNN layer are reduced to a global node scatter operation, which scatters the previous-layer activations of each node (i.e., $h^{(k-1)}_v$) along the edge direction, and a global node gather operation, which aggregates the neighbors of each node by gathering in-edges.

The updates in each GNN layer are reduce to a global node update operation, which computes the new activations of each node (i.e., $h^{(k)}_v$).

\sys can directly benefit from various existing system optimizations for graphs.
We implemented \sys on top of Lux~\cite{Lux}, a high-performance graph processing framework, and used cuDNN~\cite{cudnn} and cuBLAS~\cite{cublas} as the underlying libraries to perform tensor operations such as matrix multiplications.

%We implemented our GNN framework in Lux~\cite{Lux}, a high-performance graph processing framework.
Figure~\ref{fig:compare_training} compares the training performance of TensorFlow, DGL with PyTorch~\footnote{https://www.dgl.ai/}, and \sys and shows that \sys significantly outperforms existing frameworks that use sparse matrix operations to perform GNN training.
The speedup is due to a number of system optimizations enabled by maintaining graph structures in \sys.

It is worth noting that our framework uses the same program interface as DGL, and therefore existing GNNs can directly run on our framework without any changes to the GNNs.


\section{Experiments}
\label{sec:exp}
\begin{table}
\caption{Datasets used in the experiments.}
\label{tab:datasets}
\resizebox{\columnwidth}{!}{
\begin{tabular}{|l|l|l|l|}
\hline
{\bf Name} & {\bf \# Nodes} & {\bf \# Edges} \\
\hline
\multicolumn{3}{|c|}{Node Classification} \\
\hline
BZR~\cite{BZR} & 6,519 & 137,734\\
%SST & & & Node Classification \\
PPI~\cite{PPI} & 56,944 & 1,612,348\\
REDDIT~\cite{GraphSAGE} & 232,965 & 57,307,946\\
\hline
\multicolumn{3}{|c|}{Graph Classification}\\
\hline
IMDB~\cite{COLLAB} & 19,502 & 197,806\\
COLLAB~\cite{COLLAB} & 372,474 & 12,288,900\\
\hline
\end{tabular}
}
\end{table}

Our \xg abstraction maintains predictive performance of GNNs but leads to much faster runtime performance. 
This section evaluates the runtime performance of \xgs on five real-world graph datasets.
We evaluate \xgs along three dimensions: (a) end-to-end training performance; (b) number of aggregations; and (c) number of data transfers in GNN training.

%We perform all experiments in our framework due to its significant better performance than other frameworks (see Figure~\ref{fig:compare_impl}). We compare the runtime performance of \xgs with ordinary graphs on the following GNN models.
%{\bf GNN models.} 
%We compare the performance of hierarchical aggregation graphs with ordinary graphs on all GNN models in Table~\ref{tab:gnns}. 

{\bf Datasets.} %We evaluate the performance of \xg on five real-world datasets.
Table~\ref{tab:datasets} summarizes the datasets used in our experiments.
BZR is a chemical compound dataset, where each node is an atom and an edge is a chemical bond between two atoms~\cite{BZR}.
PPI contains a number of protein-protein interaction graphs, each of which corresponds to a different human tissue~\cite{PPI}.
REDDIT is an online discussion forum dataset, with each node being a Reddit post and each edge being commenting relations. For both PPI and REDDIT, we directly use prepossessed data from~\citet{GraphSAGE}.
IMDB and COLLAB are two collaboration datasets for graph classification~\cite{COLLAB}.
IMDB is a movie collaboration dataset, with each node representing an actor/actress, while COLLAB is a scientific collaboration dataset, with each node representing a researcher.

%{\bf Baselines.} We compare the runtime performance of \xgs with TensorFlow and DGL that uses sparse matrix operations to perform GNN operations on standard computation graphs. 

{\bf Experimental setup.} All experiments were performed on a GPU machine with a Intel 10-core E5-2600 CPU, 256G main memory, and one NVIDIA Tesla V100 GPU.
Following previous work~\cite{GCN, GraphSAGE}, each GNN model has two GNN layers and one SoftMax layer. For graph classification datasets, each GNN model also includes a mean-pooling layer to gather graph-level activations.
%We set the number of hidden dimensions to 15.
For all experiments, we set the maximum \er{capacity} of $|\m{V}_A|$ in a \xg to be $|\m{V}| / 4$, which achieves high performance on real-world graphs in the experiments.
Section~\ref{subsec:eval_para} studies how different capacities affect the runtime performance of \xgs.
In all experiments, the memory overhead to save intermediate aggregation results is negligible: intermediate nodes consume 6MB of memory in the worst case while GNN training requires more than 7GB of memory ($\sim$0.1\% memory overhead). 

\begin{figure}[t]
    \centering
    \includegraphics[scale=0.35]{figures/training_throughputs.pdf}
    %\vspace{-5mm}
    \caption{End-to-end performance comparison among TensorFlow, DGL with the PyTorch backend, \sys with GNN-graphs, and \sys with \xgs. 
    We measure the end-to-end training throughputs on a 2-layer GCN model with 16 hidden dimensions in each layer.
    The training throughputs are normalized by the TensorFlow numbers.}
    \label{fig:compare_training}
\end{figure}

\begin{figure}[t]
    \centering
    \subfloat[Set Aggregations.]{
    \includegraphics[scale=0.35]{figures/compare_unordered_aggregation.pdf}
    }
    \\
    \subfloat[Sequential Aggregations.]{
    \includegraphics[scale=0.35]{figures/compare_seq_aggregation.pdf}
    }
    \vspace{-1mm}
    \caption{Comparing the number of aggregations and amount of data transfers between GPU threads to perform aggregations (lower is better). 
    %Computation costs and data transfers to perform neighborhood aggregations on various computation graphs (lower is better).
    %The computation costs are measured by the numbers of binary aggregations. 
    The y-axes are normalized by GNN-graphs, and the last column in each figure is the geometry mean over all datasets.
    }
    %Runtime performance comparison between ordinary graphs and \xgs on different types of aggregations. 
    %The y-axis shows the relative numbers of binary aggregations involved in each graph representation.
    \label{fig:comapre_aggregation}
\end{figure}

\subsection{End-to-End Performance}
\label{subsec:eval_end}
We first compare the end-to-end training performance between GNN-graphs and \xgs.
For GNN-graphs, we also ran experiments on TensorFlow (v1.12) and DGL with the PyTorch backend (v1.0) to compare the time it takes to complete one epoch of training using different frameworks.

Figure~\ref{fig:compare_training} shows the comparison results.
By using the same GNN-graphs, \sys outperforms TensorFlow and DGL with the PyTorch backend by 3.7-5.5$\times$.
The performance improvement is achieved by a number of critical system optimizations enabled in \sys, as discussed in Section~\ref{sec:impl}.

Compared to directly training GNN-graphs in \sys, \xgs can further improve the training throughput by up to 2.8$\times$, while maintaining the same network accuracy.
We note this improvement is achieved completely automatically, and computing a \xg is inexpensive.
Thus, because the improvement is essentially for free, we believe there is no reason not to use \xgs in preference to GNN-graphs.
%The speedup is achieved by eliminating redundant computation and unnecessary data transfers in GNN computation.

%We now compare the end-to-end training performance of ordinary graphs and \xgs on two GNN models. Figure~\ref{fig:compare_training} shows the comparison results.
%Both GCN and GCN-P use unordered aggregations, and GraphSAGE-LSTM uses sequential aggregations.
%GraphSAGE-LSTM requires an ordering on each node's neighbors. For each node, we order its neighbors by their degrees. 
%Compared to directly training GNN models on ordinary graphs, \xgs maintain the same network accuracy while reducing the end-to-end training time by up to 47\%. 
%The performance improvement is achieved by eliminating redundant computation and reducing unnecessary memory accesses in neighborhood aggregations.
%This shows that \xg provides a more efficient graph representation to train GNNs.

\subsection{Aggregation Performance}
\label{subsec:eval_agg}
We further compare the aggregation performance of GNN-graphs and \xgs on the following two metrics: (1) the number of binary aggregations performed in each GNN layer; and (2) the amount of data transfers between GPU threads to perform the aggregations.

%\jure{Here you need to explain the evaluation metrics. You need to define what does it mean to count the number of aggregations and (more importantly) what is the number of data transfers. Explain precisely what do we measure.}

%We further analyze the performance of \xgs and standard graphs by comparing the computation costs and data transfers to perform neighborhood aggregations in each computation graph.
%The computation cost to perform neighborhood aggregations is measured by the number of binary aggregations involved in the graph.

%Training a GNN model on ordinary graphs and their equivalent \xgs achieve the same model accuracy, and they only differ in the neighborhood aggregation scheme.
%Therefore, we first evaluate the aggregation performance on ordinary graphs and their equivalent \xgs found by the greedy algorithm.
%We compare the computation costs and memory accesses to perform neighborhood aggregations.

Figure~\ref{fig:comapre_aggregation} shows the comparison results.
For GNNs with set aggregations, \xgs reduce the number of aggregations by 1.5-6.3$\times$ and reduce the amount of data transfers between GPU threads by 1.3-5.6$\times$. 
For GNNs with sequential aggregations, \xgs reduce the number of aggregations and data transfers by up to 1.8$\times$ and 1.9$\times$, respectively.

Although the search algorithm finds a globally optimal \xg for sequential aggregations (Theorem~\ref{thm3}) and a $(1-1/e)$-approximation of globally optimal \xgs for set aggregations (Theorem~\ref{thm4}), we observe the performance improvement is more significant for set aggregations.
Optimality for \xgs with sequential and set aggregations are of course different problems.
Because set aggregations can be reordered to eliminate more redundant aggregations, higher performance is possible for \xgs with set aggregations, though optimal solutions are more difficult to compute.
%This is because the greedy algorithm can opportunistically reorder aggregations to further eliminate redundant aggregations.

It is also worth noting that the \xg search algorithm can find highly optimized \xgs even on very sparse graphs.
For example, on the COLLAB dataset with a graph density of 0.01\%, the \xg search algorithm reduces the number of aggregations and data transfers by 3.3$\times$ and 2.2$\times$ for set aggregations, respectively.

\begin{figure}[t]
    \centering
    \includegraphics[scale=0.32]{figures/cost_model.pdf}
    \vspace{-4mm}
    \caption{Relations between the capacities of different \xgs and their per-epoch GCN training time on the COLLAB dataset. The squares show the training time of \xgs with different capacities. The red line indicates the training time of the best discovered \xg by the search algorithm.}
    \label{fig:capacity}
\end{figure}

\subsection{Capacity}
\label{subsec:eval_para}
We study how different values of capacity affect the runtime performance of the generated \xgs. 
Recall that capacity is an upper bound on the number of aggregation nodes in a \xg.
In the cost model, a larger value of capacity allows the \xg search algorithm to eliminate more redundant aggregations and therefore achieves lower cost.

Figure~\ref{fig:capacity} shows that a larger value of {\em capacity} can consistently improve the end-to-end training performance, which indicates that the cost model is an appropriate metric to evaluate and compare the performance of different \xgs.

By gradually releasing the capacity constraint, the search algorithm eventually finds a \xg with $\sim$150K aggregation nodes, which consume 6MB of memory (0.1\% memory overhead) while improving the training performance by 2.8$\times$.

\section{Related Work}
\label{sec:related}
{\bf Graph neural networks} are used to solve various real-world tasks with relational structures~\cite{CNLMF, GCN, GraphSAGE, DiffPool, GIN}.
This paper solves an orthogonal problem: how to optimize GNN efficiency while maintaining network accuracy.
The \xg representation is agnostic to any particular GNN model and provides a general approach that can be automatically applied to eliminate redundancy for arbitrary GNN models.

{\bf Computation reduction in neural networks.} A number of techniques have been proposed recently to reduce computation in neural networks. 
For example, ~\citet{Han1} presents a weight pruning algorithm to iteratively remove weak connections in a network. 
As another example, ~\citet{Han2} proposes a deep compression technique to reduce network computation by training on low precision weights.
These techniques reduce network computation at the cost of modifying computation in the network, resulting in decreased network accuracy (as reported in these papers).
By contrast, in this paper, we propose a new GNN representation that accelerates GNN training by eliminating redundant computation in GNN-graphs while maintaining the original computation and network accuracy.


\section{Conclusion}
We have introduced \xg, a new GNN graph representation that explicitly avoids redundant computation in GNNs by managing intermediate aggregation results hierarchically.
We propose a cost model to quantitatively evaluate the runtime performance of different \xgs and use a \xg search algorithm to find optimized \xgs under the cost model.
Our experiments show that \xgs significantly outperform existing GNN-graphs by improving the end-to-end training performance and reducing the aggregations and data transfers in GNN training.
%We are planning to release our implementations upon acceptance to facilitate future research.
\bibliography{bibliography}
\bibliographystyle{icml2019}

\end{document}
